\documentclass{article}

\usepackage[ngerman]{babel} 
\usepackage[T1]{fontenc}
\usepackage{amsfonts} 
\usepackage{setspace}
\usepackage{amsmath}
\usepackage{amssymb}
\usepackage{titling}


\newcommand*{\qed}{\null\nobreak\hfill\ensuremath{\square}}
\newcommand*{\puffer}{\text{ }\text{ }\text{ }\text{ }}
\newcommand*{\gedanke}{\textbf{-- }}


\pagestyle{plain}
\allowdisplaybreaks

\setlength{\droptitle}{-14em}
%\setlength{\jot}{12pt}
\setlength{\hoffset}{-3cm}
\setlength{\voffset}{-1cm}
\setlength{\textheight}{674pt}
\setlength{\textwidth}{426pt}


\title{Mathe B Klausurzettel}
\author{Henri Paul Heyden \\ \small{stu240825}}
\date{}

\begin{document}
\maketitle
Sei K Körper, zum Beispiel \(\mathbb R\).
\subsection*{Analysis}
\subsubsection*{Folgen}
Folgen sind Funktionen aus \(K ^ {\mathbb{N}_\mu}\) für \(\mu \in \mathbb{N}\) als Startindex. \\
Die Menge an Folgen nach \(K\) bezeichnen wir als \(\mathcal{S}(K)\)
\subsubsection*{Umgebungen}
Eine Umgebung von \(x\in K\) ist eine Menge an Intervallen, in denen \(x\) innerer Punkt ist. \\
Eine Umgebung von \(x\) ist die Kugel mit Radius \(\delta \in K\): \(B(x, \delta) := (x-\delta, x+\delta)\). \\
\(\mathcal{U}(x)\) ist die Menge aller Umgebungen um \(x\).
\subsubsection*{Limes}
Der Limes einer Folge ist die Zahl, für die nur endlich viele Umgebungen existieren, in denen keine Folgekomponenten liegen.\\
Es gilt für eine Folge \((x_n)_n\):\\
\(\lim_n x_n = p \in \overline{\mathbb{R}} \Longleftrightarrow \forall \epsilon > 0: \exists n_0: \forall n \ge n_0: |x_n - p| < \epsilon\) \\ \\
Sandwichsatz für die Folge \((x_n)_n\):\\
\((\exists a_n, b_n \in \mathcal{S}(\mathbb R): a_n \ge x_n \ge b_n \wedge p = \lim_{n} a_n = \lim_{n} b_n) \Longrightarrow \lim_n x_n = p\) \\ \\
Teilfolge: \\
Ist \((x_n)_n\) Folge mit Limes \(p\), dann haben alle Folgen mit \(o \in \mathbb N ^ \mathbb N\) streng monoton steigend und \(a_n := x_{o(n)}\) Limes \(p\). \\ \\
Kombinationssätze:\\
1) \(\lim_n c \cdot x_n = c \cdot \lim_n x_n\) \\
2) \(\lim_n (x_n + y_n) = \lim_n x_n + \lim_n y_n\) \\
3) \(\lim_n x_n \cdot y_n = \lim_n x_n \cdot \lim_n y_n\) \\ \\
Reihen \\
Reihen sind Folgen über Folgen. Sei \((x_n) \in \mathcal{S}(\mathbb R)\).\\ Dann bezeichnen wir die Reihe über \((x_n)_\mu\) als \(\left(\sum_{k = \mu}^{n} x_k\right)_n\) \\ \\
Wurzelkriterium:\\
Sei \((x_n)_n\) Folge. Sei \(p = \lim_n \sqrt[n]{|x_n|}\) \\
\(p < 1 \Longrightarrow \) die Reihe über \(x_n\) konvergiert absolut. \\
\(p > 1 \Longrightarrow \) die Reihe über \(x_n\) divergiert. \\ \\
Quotientenkriterium:\\
Sei \((x_n)_n\) Folge. Sei \(p = \lim_n \lim_{n}|\frac{x_n}{x_{n-1}}|\) \\
\(p < 1 \Longrightarrow \) die Reihe über \(x_n\) konvergiert absolut. \\
\(p > 1 \Longrightarrow \) die Reihe über \(x_n\) divergiert. 
\subsubsection*{Topologie}
Wir bezeichnen \(x\) einen HP, wenn \(\exists a_n \in \mathcal{S}(\mathbb{R} \setminus \{x\}): \lim_{n} a_n = x\) gilt.
\subsubsection*{Funktionslimes}
Sei \(x\) HP von \(\Omega \subseteq \mathbb R\) und \(f \in \Omega ^ \mathbb R\). Dann bezeichnen wir den Funktionslimes von \(f\) in \(x\) als \(\lim_{n} f(z_n)\) für alle \(z_n \in \mathcal S (\Omega \setminus \{x\})\) mit Limes \(x\), wenn er existiert. \\
Wir schreiben das dann auch: \(\lim_{z \rightarrow x} f(z)\). \\
Die Kombinationssätze für den Limes gelten auch für den Funktionslimes.
\subsubsection*{Stetigkeit}
Wir nennen eine Funktion stetig in \(x\), wenn \(x\) kein HP ist,\\
oder wenn \(\lim_{z \rightarrow x}f(z) = f(x)\) gilt. \\
Stetigkeit von \(f\) in \(x\) ist äquivalent zu:\\
\(\forall \epsilon > 0: \exists \delta > 0: \forall z \in B(x, \delta) \cap \Omega: f(z) \in B(f(x), \epsilon)\)
\subsubsection*{Differenzierbarkeit}
Sei \(\Omega \subseteq \mathbb R\). Wir nennen eine Funktion \(f \in \mathbb R^\Omega\) differenzierbar in einem HP \(x\),\\
wenn gilt: \(\lim_{z\rightarrow x}\frac{f(z) - f(x)}{z - x} \in \mathbb R\). \\
Differenzierbarkeit impliziert Stetigkeit. \\
Somit sind rationale Funktionen und Polynome differenzierbar und stetig. \\ \\
Kombinationssätze für Differenzierbarkeit: \\
1) \((c\cdot f')(x) = c\cdot f'(x)\) \\
2) \((f + g)'(x) = f'(x) + g'(x)\) \\
3) \((f\cdot g)'(x) = f'(x)\cdot g(x) + f(x)\cdot g'(x)\) \\
4) \(\left(\frac{1}{f}\right)'(x) = \frac{-f'(x)}{f(x)^2}\) \\
5) \(\left(\frac{f}{g}\right)'(x) = \frac{f'(x)\cdot g(x) - f(x) \cdot g'(x)}{g(x)^2}\) \\ \\
Kettenregel: \((g \circ f)'(x) = g'(f(x))\cdot f'(x)\)
\subsubsection*{Monotonie und lokale Extremstellen}
Sei \(I\) echtes Intervall und \(f \in \mathbb R ^I\) in \(I_0\) differenzierbar. Dann gilt:\\
1) \((\forall x \in I_0: f'(x) = 0) \Longleftrightarrow \) f ist konstant. \\
2) \((\forall x \in I_0: f'(x) > 0) \Longrightarrow \) f ist streng monoton steigend. \\
3) \((\forall x \in I_0: f'(x) < 0) \Longrightarrow \) f ist streng monoton fallend. \\ \\
Wir bezeichnen \(LMAX(f), LMIN(f), LEXT(f)\) die Mengen an die Stellen der lokalen Maxima, lokalen Minima und Extremstellen.
\end{document}